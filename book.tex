Regions
The SkRegion class contains an Iterator class, which can be used to traverse the rectangles that compose a region object.
This Iterator class contains four fields, all private: fRgn, fRuns, fRect, fDone. fRgn can be accessed through the rgn() getter; fRect through rect() and fDone  through fDone.
There are two constructor. The first takes no arguments and, actually, doesn't return an usable Iterator (since it is associated to Region). Such an Iterator needs a call to reset before being usable. The second constructor, instead, take a const SkRegion& parameter; the only thing it does anyway is calling reset().
reset sets fRgn to rgn; it then checks for the run type of rgn and takes different decision in the three cases.
If rgn is empty, it sets fDone to true (nothing to traverse!) and returns.
If rgn is a rect, fDone is false (still one rectangle to traverse), fRect set to rgn.fBounds and fRuns to nullptr.
In the remaning case, if rgn is complex, fDone is false, fRect is ignored (not set to anything) and fRuns is set in a way we'll explain in a moment.
